\documentclass{article}

\usepackage[latin1]{inputenc}
\usepackage[T1]{fontenc}
\usepackage{textcomp}
\usepackage{mathptmx}
\usepackage[scaled=.92]{helvet}
\usepackage{courier}
\renewcommand*{\familydefault}{phv}
\usepackage[margin=1in]{geometry}
\usepackage{fancyhdr}
\lhead{ABC, Inc.}\rhead{XYZ Dept.}
\pagestyle{fancy}
\usepackage{graphicx}
\usepackage{color}
\usepackage[
  pdftex,
  colorlinks=true,
  pdftitle={Key form},
  pdfsubject={Key},
  pdfauthor={ich},
  pdfpagemode=UseNone,
  pdfstartview=FitH,
  pagebackref,
  pdfhighlight={/N}
]{hyperref}

\newcommand{\textforlabel}[2]{%
  \TextField[name={#1},value={#2},width=7em, align=2, bordercolor={1 1 1},
             readonly=true]{}
}

% \begin{insDLJS}[latex-identifier]{latex-identifier}{comment} .. \end{insDLJS}
%   creates an environment for defining javascript code
% \OpenAction{/S /JavaScript /JS (jsFunctionCall();)}
%   defines what happens every time someone goes to page 1
\usepackage[pdftex]{insdljs}  % for inserting javascript \begin{insDLJS}

\begin{insDLJS}[exaaaa]{exbbbb}{Document Level Javascript}
  // indicate that the function was not yet run
  var done = 0;
  
  // function to run when opening the document
  function myFirstJavaScriptFunction()
  {
    if (!done) {
      done = 1;
      app.alert("The form was opened.");
    }
  }
\end{insDLJS}
\OpenAction{/S /JavaScript /JS (myFirstJavaScriptFunction();)}

\begin{document}
\section*{Request for a key}
\begin{Form}
  \begin{tabular}{|rl|}
    \hline
    & \\*[-0.9em]
    \multicolumn{2}{|c|}{\textbf{Employee}} \\

    \textforlabel{vn}{First name:}
      & \TextField[name=vorname, width=20em, bordercolor={0.65 0.79 0.94}]{} \\
    \textforlabel{nn}{Name:}
      & \TextField[name=name, width=20em, bordercolor={0.65 0.79 0.94}]{} \\
    \textforlabel{ab}{Department:}
      & \ChoiceMenu[name=abt, width=20em, combo=true, bordercolor={0.65 0.79 0.94}]{}{%
          Sales=v,
          Production=f,
          Services=s} \\
    \textforlabel{pp}{Picture:}
      & \TextField[name=pic, width=20em, fileselect=true, bordercolor={0.65 0.79 0.94}]{} \\

    \hline
    & \\*[-0.9em]
    \multicolumn{2}{|c|}{\textbf{Time}} \\

    \textforlabel{z}{Time:}
      & \ChoiceMenu[name=zeit, width=20em, combo=true, bordercolor={0.65 0.79 0.94}]{}{%
          limited=b,
          unlimited=u} \\
    \textforlabel{v}{From:}
      & \TextField[name=from, width=10em, bordercolor={0.65 0.79 0.94}]{} \\
    \textforlabel{b}{Until:}
      & \TextField[name=until, width=10em, bordercolor={0.65 0.79 0.94}]{} \\

    \hline
    & \\*[-0.9em]
    \multicolumn{2}{|c|}{\textbf{Doors}} \\

    \textforlabel{th}{Front door:}
      & \CheckBox[name=ht, width=1.2em, bordercolor={0.65 0.79 0.94}]{} \\
    \textforlabel{t1}{Ground floor:}
      & \CheckBox[name=e1, width=1.2em, bordercolor={0.65 0.79 0.94}]{} \\
    \textforlabel{t2}{First floor:}
      & \CheckBox[name=e2, width=1.2em, bordercolor={0.65 0.79 0.94}]{} \\

    \hline
  \end{tabular}
\end{Form}
\end{document}
