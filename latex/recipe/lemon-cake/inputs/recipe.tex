\title{Colton Jackson's Lemon Cake}
\PrepTime{40}
\CookingTempe{350}
\CookingTime{25-30}
\TypeCooking{Dessert}
\NbPerson{6}
\Image{0 650 1200 750}{images/LEMON-LAYER-CAKE-2.jpg}

\begin{ingredient}
  \begin{main}
    \item 1 cup (226 g) unsalted butter, room temperature
    \item 1/3 cup (72.6 g) canola oil
    \item 1 teaspoon (5.6 g) salt
    \item 1 3/4 cups (350 g) granulated sugar
    \item 5 whole eggs, room temperature
    \item 2 eggs yolks, room temperature
    \item 3 cups (345 g) cake flour
    \item 2 teaspoons (8 g) baking powder
    \item 1 cup (240 g)  sour cream, room temperature
    \item 1 teaspoon (4.2 g)  clear vanilla extract 
  \end{main}
  \begin{subingredient}{Lemon Curd}
    \item 1 cup (200 g) granulated sugar
    \item 1 tablespoon (9 g) grated lemon zest one large lemon
    \item 1/2 cup lemon juice (about 3 lemons)
    \item 3 large eggs
    \item 4 egg yolks
    \item 4 tablespoons (56.5 g) cold unsalted butter cut into small pieces
  \end{subingredient}
  \begin{subingredient}{Frosting}
    \item 2 Cups softened butter
    \item 5 cups powdered sugar
    \item 1/4 cup heavy cream
    \item 1 tablespoon vanilla
    \item pinch of salt
  \end{subingredient}
  \begin{subingredient}{Drip}
    \item 1 cup white chocolate or dark chocolate chips
    \item 3/4-1 cup heavy cream
  \end{subingredient}
\end{ingredient}
\begin{recipe}
  \step{
    Preheat the oven to 350 degrees F. Spray three 8-inch or four 6-inch round
    cake pans with non-stick spray, add parchment and another coat of spray.
    Set aside.
    }
  \step{
    In a medium size bowl, sift the cake flour and baking powder. Set aside.
    }
  \step{
    In the bowl of a stand mixer, fitted with the paddle attachment, beat the
    butter, canola oil, salt and sugar on high until fluffy and smooth, about 5
    minutes.
    }
  \step{
    Add the eggs and egg yolks, one at a time to the bowl and beat on medium
    until well incorporated.
    }
  \step{
    Turn the mixer on low and add 1/3 of the flour mixture to the butter
    mixture and mix until fully incorporated. Then add half of the sour cream
    and blend well, followed by another 1/3 of the flour mixture, the second
    half of the sour cream and lastly, add the remaining flour mixture and mix
    until incorporated.
    }
  \step{
    Add the clear vanilla extract and butter extract and beat on low until
    incorporated. Scrape down the sides of the bowl and mix on low for another
    30 seconds.
    }
  \step{
    Evenly distribute the cake batter among the three pans and bake for 25-30
    minutes or until a toothpick inserted into the middle of the cakes comes
    out clean or with just a few crumbs. Make sure not to over bake.
    }
  \step{
    Remove cakes from the oven and let cool in pans for 10 minutes before
    inverting onto cooling racks.
    }
  \step{
    Once completely cooled, wrap in plastic wrap and place in the freezer or
    refrigerator to chill before frosting.
    }
  \substep[Lemon Curd]{
    Whisk the sugar, lemon zest, and lemon juice in a medium saucepan. Whisk in
    the whole eggs and yolks in a small bowl and then whisk them into the lemon
    mixture.
    }
  \substep[Lemon Curd]{
    Cook the mixture over medium-low heat, whisking constantly, until it's
    thick like pudding, about 6 to 8 minutes. Remove from the heat and whisk in
    the butter, a few pieces at a time until incorporated.
    }
  \substep[Lemon Curd]{
    Strain the mixture through a fine-mesh sieve into a small bowl, pushing it
    through with a rubber spatula. Press plastic wrap directly onto the surface
    to prevent a skin from forming. Refrigerate until completely set, at least
    4 hours and up to 5 days.
    }
  \step{\textbf{Frosting:} combine all ingredients in a separate bowl}
  \step{\textbf{Drip:} combine all ingredients in a separate bowl}
\end{recipe}

%FOR THE CAKE
%1 cup (226 g) unsalted butter, room temperature
%1/3 cup (72.6 g) canola oil
%1 teaspoon (5.6 g) salt
%1 3/4 cups (350 g) granulated sugar
%5 whole eggs, room temperature
%2 eggs yolks, room temperature
%3 cups (345 g) cake flour
%2 teaspoons (8 g) baking powder
%1 cup (240 g)  sour cream, room temperature
%1 teaspoon (4.2 g)  clear vanilla extract 
%
%Preheat the oven to 350 degrees F. Spray three 8-inch or four 6-inch round cake pans with non-stick spray, add parchment and another coat of spray. Set aside.
%In a medium size bowl, sift the cake flour and baking powder. Set aside.
%In the bowl of a stand mixer, fitted with the paddle attachment, beat the butter, canola oil, salt and sugar on high until fluffy and smooth, about 5 minutes.
%Add the eggs and egg yolks, one at a time to the bowl and beat on medium until well incorporated.
%Turn the mixer on low and add 1/3 of the flour mixture to the butter mixture and mix until fully incorporated. Then add half of the sour cream and blend well, followed by another 1/3 of the flour mixture, the second half of the sour cream and lastly, add the remaining flour mixture and mix until incorporated.
%Add the clear vanilla extract and butter extract and beat on low until incorporated. Scrape down the sides of the bowl and mix on low for another 30 seconds.
%Evenly distribute the cake batter among the three pans and bake for 25-30 minutes or until a toothpick inserted into the middle of the cakes comes out clean or with just a few crumbs. Make sure not to over bake.
%Remove cakes from the oven and let cool in pans for 10 minutes before inverting onto cooling racks.
%Once completely cooled, wrap in plastic wrap and place in the freezer or refrigerator to chill before frosting.
%
%FOR THE LEMON CURD
%1 cup (200 g) granulated sugar
%1 tablespoon (9 g) grated lemon zest one large lemon
%1/2 cup lemon juice (about 3 lemons)
%3 large eggs
%4 egg yolks
%4 tablespoons (56.5 g) cold unsalted butter cut into small pieces
%Whisk the sugar, lemon zest, and lemon juice in a medium saucepan. Whisk in the whole eggs and yolks in a small bowl and then whisk them into the lemon mixture.
%
%Cook the mixture over medium-low heat, whisking constantly, until it's thick like pudding, about 6 to 8 minutes. Remove from the heat and whisk in the butter, a few pieces at a time until incorporated.
%
%Strain the mixture through a fine-mesh sieve into a small bowl, pushing it through with a rubber spatula. Press plastic wrap directly onto the surface to prevent a skin from forming. Refrigerate until completely set, at least 4 hours and up to 5 days.
%
%For the frosting:
%
%2 Cups softened butter
%
%5 cups powdered sugar
%
%1/4 cup heavy cream
%
%1 tablespoon vanilla
%
%pinch of salt
%
%
%For the Drip:
%
%1cup white chocolate or dark chocolate chips
%
%3/4-1 cup heavy cream





%%------------------------------------------
%% information doc
%\title{Florentins au chocolat}
%\PrepTime{50}
%\CookingTime{10}
%\CookingTempe{180}
%\TypeCooking{Plaque}
%\NbPerson{4}
%\Image{0 100 400 200}{images/florentin} %style 1
%%\Image{0 100 800 900}{images/lasagnes} %style 2
%%------------------------------------------
%
%\begin{ingredient}
%%\vspace{0.5cm}
%\begin{main}
%	\item 4 cà.S de sucre en poudre
%	\item 1 cà.S de crème liquide
%	\item 1 cà.S de miel
%	\item 1 grosse noix de beurre
%	\item 35 gr d’amandes effilées
%	\item 50 gr de chocolat au lait ou chocolat blanc
%\end{main}
%\begin{subingredient}{Test subingredient}
%	\item 1 cà.c de test1
%	\item 1 à 2 cà.S de test2
%	\item 3 gouttes de test3
%	\item 8 morceaux de test4.	
%\end{subingredient}
%\end{ingredient} %no space with \begin{recipe}
%\begin{recipe}
%\step{Préchauffez votre four à 180°C (th.6).}
%\step{Dans une casserole, faites bouillir le sucre en poudre avec la crème liquide, le beurre et le miel.}
%\step{Une fois que le sucre prend une jolie coloration brune, versez les amandes effilées dans la casserole, et remuez bien pour napper l’intégralité des amandes.}	
%\step{Pour la cuisson au four vous avez 2 possibilités : Soit vous versez la « pâte » dans le fond de moules en silicone, type moules à muffins ou moules à tartelettes, soit vous étalez bien la « pâte », et rapidement car le caramel durcit vite, sur la plaque de votre four recouverte d’une feuille de papier sulfurisé, et après la cuisson vous découperez des cercles à l’aide d’un emporte-pièces rond.}
%\step{Dans tous les cas, mettez la « pâte » au four pendant 3 à 5 minutes. A la sortie du four, soit vous découpez tout de suite des ronds à l’aide de l’emporte-pièces, soit vous laissez refroidir les florentins avant de les démouler de vos moules à muffins.}
%\step{Pendant que les florentins refroidissent, faites fondre le chocolat au lait ou blanc soit au bain-marie, soit au micro-ondes à faible puissance, soit dans une petite casserole à feu doux.}
%\step{Trempez ensuite la moitié des florentins dans le chocolat fondu et mettez-les au réfrigérateur pendant une bonne vingtaine de minutes pour que le chocolat prenne bien.}
%\substep[Test substep]{Blabla}
%\substep{Blabla}
%
%\end{recipe}
%
%\begin{notes}
%
%\end{notes}	
